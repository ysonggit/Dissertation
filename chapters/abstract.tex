The dissertation first presents our geometric-based planning algorithm
for robots to describe and reason about
uncertainty, termed the ``constrained geometric approximation approach''.

Previous work delivered computationally expensive methods to represent a robot's
knowledge of its states and uncertainty explicitly using information state in
the information space. 
%
In contrast, our approach is computationally inexpensive.
Specifically, this approximation method is suitable for the robots 
with extreme limitations in both sensing and computation. 
%
We provided simulations of sensor-based navigation
tasks, along with experimental results.
%
We conclude that the robot can achieve similar success rate at a small fraction of the computational cost,
by maintaining simple representations of the incomplete information.
  
The dissertation also contributes two types of decentralized algorithms for the multi-robot
lattice pattern formation problem. 
%
Both of the formation algorithms feature in representing the lattice as
a directed graph, in which edges show the desired rigid body transformations
between the local frames of pairs of neighbor robots. 
%
Another common feature of the two algorithms is that robots organize themselves using tree structures. 

The primary formation algorithm enables each robot to perform distributed task assignment using
only local information.
%
The experimental results demonstrate that the algorithm executes efficiently and 
scales reasonably well as the number of robots increases.
Moreover, the algorithm is robust against the robots' failures.


With regard to the limitations of the task-assignment-based algorithm, 
we have designed another formation algorithm that has new features involving:
the bounded execution time, a novel motion strategy that guarantees the network connectivity, 
and the improved final lattice formation quality.
%
Furthermore, we have proved the correctness of the new algorithm and 
conducted groups of experiments to compare the timing performance 
and the formation qualities of both formation algorithms.
